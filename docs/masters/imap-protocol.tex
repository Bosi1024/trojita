% vim: spelllang=en spell textwidth=120
\documentclass[trojita]{subfiles}

\begin{document}

\chapter{IMAP Protocol Introduction}
\label{sec:imap-protocol}

\begin{abstract}
This chapter provides a gentle introduction to peculiarities of the IMAP protocol.  We also present detailed analysis of
the available extensions and how they can be used to improve the overall IMAP experience.
\end{abstract}

\section{IMAP}

The IMAP protocol, as defined by RFC~3501~\cite{rfc-imap}, is an internet protocol suitable for managing e-mail folders
and individual messages stored on a remote mail server.  In contrast to the older POP3 protocol~\cite{rfc-pop3}, IMAP is
really intended to serve as an {\em access} protocol.  Where a POP3 client would happily download a full message from
the mail server, store it into a local mailbox and perform all further processing locally, the IMAP mode of operation is
much more complicated.  These complications, however, bring a whole slew of new features and interesting applications
along.

For one, IMAP presents a single, authoritative place storing messages -- that feature alone is a must in today's world
where people are expecting to be able to access their mail from their cell phones.  Furthermore, given that all messages
are located on a single place, it is possible to perform efficient, server-side operations over the whole mailstore,
like searching or sorting.  IMAP also makes it possible to access individual message parts like attachments separately,
eliminating the need to download a huge message before reading a short accompanying textual information.  Finally,
advanced servers can recognize clients with limited resources and only present a subset of messages to them.

At the same time, IMAP is an old protocol burdened by many compatibility warts.  Its designers were struggling with
people objecting to novel ideas due to legacy code in their mail implementations.  Over the years, though, various
protocol extensions appeared.  Some of them are extremely useful for contemporary clients, yet they cannot be relied on
because there is no general agreement on what extensions are really crucial, and hence available in most of IMAP
servers.

The rest of this chapter provides a quick overview of the basic IMAP concepts and how they relate to the usual client's
workflow.

\subsection{Basic Features}

An IMAP server exports a set of {\em mailboxes}, folders which contain individual messages (and further mailboxes, if
the server allows that).  Each message can be identified either by its {\em sequence number}, an order in which it
appears in mailbox, or by its {\em UID}.  Sequence numbers are by definition very volatile (deleting the first message
in a mailbox changes sequence numbers of all subsequent messages) while the UIDs provide better chances of persistence
across reconnects.~\footnote{It shall be noted that IMAP does {\em not} guarantee UIDs to be persistent at all.  The
reason behind this decision was to allow IMAP to publish messages from obsolete mail stores which could not have been
extended to support UIDs at all.  Even today, UID changes have to be expected when signalled through {\tt UIDVALIDITY}.}
When the UIDs have to be invalidated for some reason, a per-mailbox integer property {\tt UIDVALIDITY} shall be
incremented to signal its clients that previously used UIDs are no longer valid.

\subsection{Cache Filing Protocol}

As Mark Crispin, the principal author of the IMAP standard, has to say~\cite{crispin-imap-cache-filing}, IMAP is a {\em
cache filing protocol}.  That means that whatever the server thinks about a mailbox state is {\em the truth}, and any
state stored on the clients can be invalidated by the server at any time.  This critical design choice has impact on all
further operations.  IMAP clients which do not anticipate such a behavior~\footnote{Such clients are usually called
``POP3 clients converted to speak IMAP'' on various IMAP-related mailing
lists.~\cite{shannon-imap-clients-glorified-imap}} are bound to operate in an inefficient manner or fail in unexpected
scenarios.

The first issue which typically comes up on the {\tt imap-protocol} mailing list is treating UIDs as a persistent
identifier of some kind.  In fact, IMAP guarantees that a triple of (mailbox name, {\tt UIDVALIDITY}, {\tt UID}) will
not refer to {\em any other} message at any time, but there's no guarantee that the very same message, quite possibly in
the same mailbox, will not get another UID in future.~\footnote{People have been trying to solve this issue for quite
some time, but no standardized solution is ready yet.  The recent iterations of these proposals concentrate on providing
a cryptographic hash of a message body, but is far from clear whether doing so would get any traction.  Furthermore, the
hashes are typically too long to serve as the only identifier of a message, so UIDs will definitely be around in
future.}  That said, on reasonable server implementations, the UIDs should not get invalidated too often under normal
circumstances.  Given the IMAP protocol doesn't offer anything else, they are widely used (along with the {\tt
UIDVALIDITY} and when limited to the scope of a single mailbox) as a semi-persistent identification of a message.

UIDs are assigned to the messages in a strictly monotonic sense, i.e. if message $A$ has a sequence number $seq_A$ and
message $B$ has sequence number $seq_B$ such as $seq_A < seq_B$, it is true that $UID_A < UID_B$.  UID numbers are also
guaranteed to never be reused in the same mailbox, unless the {\tt UIDVALIDITY} changes.

Due to the facts described above, virtually any IMAP clients which maintain a persistent cache of the downloaded data
use UIDs to assign the cached data to individual messages.  Such an approach leads to a need to maintain a mapping
between the sequence numbers and the UID numbers of all messages in the mailbox -- upon a reconnect, clients have to
recognize whether any messages previously available in the mailbox disappeared, and if they did, the clients should
remove the cached data for these messages.~\footnote{The reader shall be reminded that IMAP is a {\em cache filing
protocol}, i.e. the server is always right about what messages ``are'' in a mailbox and what messages are gone.}  This
is in a strong contrast to the usual POP3 mode of operation where the clients are expected to prune their cache only
based on their local policy, perhaps moving older messages to a designated archive, but definitely not discarding the
retrieved data as soon as the server doesn't present the message anymore.  Furthermore, even during an established
session the IMAP server informs about messages being permanently deleted through the {\tt EXPUNGED} response which
contains sequence number only.  Given that the cache is usually addressed by UID, a caching client shall maintain full
UID mapping at any time.

\section{Mailbox Synchronization}
\label{sec:imap-mailbox-sync}

When an IMAP client opens a mailbox, the server provides it with a few data points about the state of the mail store.
Among this data, there's a number representing the total amount of messages in a mailbox through the {\tt EXISTS}
response, the current {\tt UIDVALIDITY} value and, finally, the {\tt UIDNEXT} which represents the lowest UID which the
next arriving message could possibly get assigned.  Please note that the {\tt UIDNEXT} is merely a lower bound of the
future UID; there is no guarantee that a message with such UID would exist in the mailbox in future.

Having these three values, the client can obtain a few optimizations before it proceeds to fetch an updated UID mapping
from the IMAP server:

\begin{itemize}
  \item If the {\tt UIDVALIDITY} has changed, the client is obliged to completely purge any data which it might have
    accumulated in its local persistent cache.
  \item If {\tt UIDNEXT} is not available, we have to resort to asking for the whole UID mapping from scratch.
  \item If the {\tt UIDNEXT} has decreased, the IMAP server exhibits a bug.  This situation is explicitly forbidden by
    the IMAP standard.  Trojitá will, nevertheless, try to work in this scenario by purging its cache and continuing as
    if no state was cached locally.
  \item If the {\tt UIDNEXT} has not changed since the last time the client has opened a mailbox, the IMAP protocol says
    that no messages could have been delivered to the mailbox at all.
    \begin{itemize}
      \item If the {\tt EXISTS} remains constant as well, it is clear that no deleted have taken place. This means that
        the cached sequence $\rightarrow$ UID mapping from the last time is directly usable, and the UID syncing phase
        of the mailbox synchronization is concluded.
      \item Otherwise, if the {\tt EXISTS} has grown, we are talking to a non-compliant IMAP server which failed to
        adjust either {\tt UIDNEXT} or {\tt UIDVALIDITY}, and cannot assume anything about the server's behavior.
        Trojitá will gracefully degrade to a complete UID mapping resynchronization.
      \item If the {\tt EXISTS} has decreased, we can be sure that some messages have been deleted.  In this situation,
        we have two possible options on how to proceed:
        \begin{itemize}
          \item We can try to perform a binary search in the list of messages to find the first deleted message and ask
            for UIDs of all messages at the subsequent positions.  This is a heuristics which relies on an observation
            that it is more likely for users working with big mailboxes to delete messages at the end of the mailbox.
            However, each step in this incremental search requires a complete round trip to the IMAP server over a
            network; with a mailbox with tens of thousands of messages, this could lead to 17 round trips.  Given that
            real-world cellular networks like the GPRS/EDGE infrastructure, unfortunately still common in the Czech
            republic, exhibit the RTT latencies which can often be larger than one second~\cite{gprs-rtt-report}, such
            an approach to incremental synchronization of the UID mapping will have severe impact on the total
            synchronization time.
          \item Another way is to give up on possible bandwidth reduction possibility and fetch the complete UID
            mapping.
        \end{itemize}
    \end{itemize}
  \item If the {\tt UIDNEXT} has grown, some messages might have arrived to the mailbox.  There's no guarantee that any
    of them are still present, though, so the clients could use another set of heuristics:
    \begin{itemize}
      \item If the increase in {\tt EXISTS} is exactly the same as the growth of the {\tt UIDNEXT}, all of the new
        arrivals are still present in the mailbox and no message have been expunged since the last time.  The client can
        ask only for UIDs of the new arrivals.
      \item In any other case, the situation is very similar to a changed {\tt EXISTS} with constant {\tt UIDNEXT} and
        the same possible optimization about the binary search might apply.  Alternatively, clients could fetch a
        complete UID mapping.
    \end{itemize}
\end{itemize}

If the decisions described above suggest that at least a part of the UID mapping shall be updated, an IMAP client can --
in absence of the optional extensions -- use one of the following ways to update the map.  The first one is through the
generic {\tt FETCH} command:

\begin{minted}{text}
  C: y1 UID FETCH 1:* (UID)
  S: * 1 FETCH (UID 123)
  S: * 2 FETCH (UID 125)
  S: * 4 FETCH (UID 127)
  S: * 3 FETCH (UID 126)
  S: y1 OK Fetched
\end{minted}

This command simply requests the {\tt FETCH} response containing UID for each and every message in the mailbox.  The
sample results show that the received data are in no particular order and demonstrate that the UID range is not
necessary continuous.  If the heuristics shows that there is just a subset of messages with unknown UIDs, the sequence
range (the {\tt "1:*"} string in the example above) shall be changed to only refer to the relevant subset, like the {\tt
"last\_uidnext:*"}.  It is also possible to request {\tt FLAGS} (which will be described later on) at this point.

Alternatively, the {\tt UID SEARCH} command can be used as follows:

\begin{minted}{text}
    C: y1 UID SEARCH UID ALL
    S: * SEARCH 123 125 127 126
    S: y1 OK search completed
\end{minted}

As we can see, the {\tt SEARCH} response is much more compact.  In practice, the bandwidth saving is slightly lower as
the UID discovery and {\tt FLAGS} synchronization can be merged into a single {\tt FETCH} command, but the overhead is
still at least four bytes for each message in the mailbox,~\footnote{If the {\tt FLAGS} are fetched as well, the real
overhead is just the {\tt "UID<space>"} string -- the number and its trailing space is present in the {\tt SEARCH}
response as well and the overhead of the {\tt FETCH} response format is required for the updated flags anyway.} which
leads to at least 200~kB of useless data on a mailbox with fifty thousands of messages.

\subsection{Message Flags}

We have mentioned message flags when describing the mailbox synchronization.  Strictly speaking, asking for message
flags of all messages in a mailbox is not necessary provided the program is capable of lazy-loading -- flags could, for
example, only be fetched for those messages which are immediately visible on screen (probably with some intelligent
preload for those which will likely get to the viewport in near future), avoiding a potentially expensive operation.  On
the other hand, contemporary user agents typically have to display an aggregated statistics like ``$X$ unread messages,
$Y$ total'' to the user.  IMAP certainly has methods for delivering such statistics, however, the baseline
specification's only two ways of conveying are through the {\tt STATUS} command or via an explicit {\tt SEARCH}.

%\todo[inline]{write me}

\subsection{Immutable Data}

%\todo[inline]{write me}

% crispin-imap-cache-filing
% http://mailman2.u.washington.edu/pipermail/imap-protocol/2010-June/001144.html
% http://mailman2.u.washington.edu/pipermail/imap-protocol/2010-June/001156.html

% shannon-imap-clients-glorified-pop
% http://mailman2.u.washington.edu/pipermail/imap-protocol/2006-September/000261.html

% gprs-rtt-report
% http://www.netlab.tkk.fi/tutkimus/pannet/publ/rtt-report.pdf

\end{document}
