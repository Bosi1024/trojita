% vim: spelllang=en spell textwidth=120
\documentclass[trojita]{subfiles}

\begin{document}

\chapter{The Mobile IMAP}
\label{sec:mobile-imap}

Many of the existing IMAP extensions discussed in \secref{sec:imap-extensions} have the potential of improving the
client's operation tremendously.  At the same time, experience has shown that there is a certain chicken-and-egg problem
with new proposals where server vendors are not willing to invest their time into promising extensions which no client
supports yet and clients are not interested in implementing extensions which they cannot test for interoperability.  In
this chapter, I am trying to provide a concise summary of individual merit of these extensions.

\section{Comparison with the Lemonade Profile}
\label{sec:lemonade-comparison}

The Lemonade profile, as defined in RFC 4550 \cite{rfc4550} in 2006 and later updated through RFC 5550 \cite{rfc5550}
during 2009, provides a list of extensions considered ``critical'' for any mobile IMAP e-mail client.  The set of
mandatory extensions is rather big, though, and to the best of my knowledge, there is {\em no} server on the market
implementing all of the compulsory features.  One might therefore wonder what were the reasons for this lack of general
availability of the Lemonade extension family.

\section{Cross-Service Requirements}

One unique feature of Lemonade is the possibility to {\em forward messages without their prior download}.  The three
ESMTP~\cite{rfc5321} and IMAP extensions, often referred to as the {\em Lemonade trio}, namely the {\tt CATENATE}, {\tt
URLAUTH} and {\tt BURL}, allow the clients to compose a message using existing parts available from the IMAP mail store,
provide a way of generating single-purpose ``pawn tickets'' for making the composed messages available to the submission
server, and replacing the {\tt DATA} SMTP command with a way of downloading the message from the IMAP server,
respectively.

\todo[inline]{compare with Lemonade}

\section{Other Implementations}
\todo[inline]{Compare with Andorid, Nokia's QMF, Apple iOS -- that's gonna be fun}

\end{document}
