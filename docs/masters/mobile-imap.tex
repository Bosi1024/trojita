% vim: spelllang=en spell textwidth=120
\documentclass[trojita]{subfiles}

\begin{document}

\chapter{The Mobile IMAP}
\label{sec:mobile-imap}

Many of the existing IMAP extensions discussed in \secref{sec:imap-extensions} have the potential of improving the
client's operation tremendously.  At the same time, experience has shown that there is a certain chicken-and-egg problem
with new proposals where server vendors are not willing to invest their time into promising extensions which no client
supports yet and clients are not interested in implementing extensions which they cannot test for interoperability.  In
this chapter, I am trying to provide a concise summary of individual merit of these extensions.

\section{Comparison with the Lemonade Profile}
\label{sec:lemonade-comparison}

The Lemonade profile, as defined in RFC 4550 \cite{rfc4550} in 2006 and later updated through RFC 5550 \cite{rfc5550}
during 2009, provides a list of extensions considered ``critical'' for any mobile IMAP e-mail client.  The set of
mandatory extensions is rather big, though, and to the best of my knowledge, there is {\em no} server on the market
implementing all of the compulsory features.  One might therefore wonder what were the reasons for this lack of general
availability of the Lemonade extension family.

\subsection{Cross-Service Requirements}

One unique feature of Lemonade is the possibility to {\em forward messages without their prior download}.  The three
ESMTP~\cite{rfc5321} and IMAP extensions, often referred to as the {\em Lemonade trio}, namely the {\tt CATENATE}, {\tt
URLAUTH} and {\tt BURL}, allow the clients to compose a message using existing parts available from the IMAP mail store,
provide a way of generating single-purpose ``pawn tickets'' for making the composed messages available to the submission
server, and replacing the {\tt DATA} SMTP command with a way of downloading the message from the IMAP server,
respectively.  This feature prevents having to transfer potentially huge data over the network three times --- once when
the users wants to read it, second time when the message is saved to the sent folder, and finally when delivering via
SMTP.

Unfortunately, a big problem with said approach is the fact that it mandates collaboration across different services ---
an explicit trust path between the IMAP and ESMTP servers have to be set up, which is a process prone to errors
\cite{qmf-fastmail-burl-bug}.  This matter is also complicated by the fact that no open source MTA~\footnote{Mail
Transfer Agent, typically an SMTP or ESMTP server} ships with official support for {\tt BURL}.~\footnote{Unofficial
patches exists for Postfix dating back to 2010 \cite{apple-postfix-burl}, but they have not been integrated into the
mainline version as of July 2012 (the {\tt postfix-2.10-20120715.tar.gz} development snapshot.}  Situation is better on
the IMAP server front with Cyrus supporting the {\tt URLAUTH} and {\tt CATENATE} extensions out-of-box with Dovecot's
support scheduled for its upcoming 2.2 release \cite{imap-server-extension-matrix}.

\subsection{Complicated Extensions}

Some of the extensions whose support is mandated by the Lemonade proposal seems to be notoriously hard for the server
vendors to implement.

A perfect example is the {\tt CONTEXT=SORT} extension \cite{rfc5267}.  As a client developer, I recognize its extreme
usefulness and appreciate its design.  Availability of such an extension would make it extremely easy to implement
live-updated sorting in my Trojitá (and Trojitá {\em does} make use of the sort context extension).  That said, given
that no IMAP server which I am aware of announces its availability, clients have to deal with the status quo in the
meanwhile.

The {\tt CONVERT} extension \cite{rfc5259} belongs to a similar category --- the features it offers, like the
server-side downscaling of JPEG images, would be {\em very} handy on a cell phone, yet no IMAP server known to the
author includes that functionality.

Both of these RFCs were published four and five years ago, respectively, and were designed by engineers working for an
IMAP server vendor.  One cannot therefore dismiss them altogether as a product of people not having any say in the
server development.  My opinion is that the allocation of engineering resources required for shipping a particular
feature in a finished product is based on another criteria than the research activity.

\section{State of Other Client Implementations}
\todo[inline]{Compare with Andorid, Nokia's QMF, Apple iOS -- that's gonna be fun}

\end{document}
