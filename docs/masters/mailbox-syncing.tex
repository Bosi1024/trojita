% vim: textwidth=120 spell spelllang=en

\section{Mailbox Synchronization}

In the IMAP mailbox concept, the server provides an authoritative answer about the data contained in any mailbox.  The
whole job of a client is to maintain a cache of this state and keep it updated.  Towards that end, the IMAP protocol
defines a set of messages which can be exchanged between the server and the client to make sure that the client's idea
of a mailbox state eventually converges to the real state, as observed on the server.

IMAP is built around two sets of numeric identifiers --- the mailbox sequence numbers and the message UIDs.  The
sequence numbers are always strictly sequential and represent the ``order'' of a message in a mailbox.  A message's
sequence number can change during its lifetime (and indeed during an active IMAP session).

The message sequence numbers cannot serve as persistent identifiers, though.  This purpose is (more or less) fulfilled
by the unique identifiers (UIDs).  These unsigned 32bit numbers are monotonous in any given mailbox and ``should not
change'' during the lifetime of a message.~\footnote{In the rare circumstances where the server cannot provide the usual
IMAP guarantees about data immutability, the {\tt UIDNEXT} of a mailbox can be changed to force invalidation of any
cached state at the client.}  The UIDs can in turn be used by client to reasonably accurately identify a message at
(almost) any time, including when the client is working offline.

Unfortunately, the bare IMAP protocol reports certain mailbox state changes, most notably message deletion, using just
the message sequence number.  Therefore, any client which caches message data between session has to maintain an
accurate UID-sequence mapping at all times, as failure to do so would lead to stale data in the cache.  \trojita is an
IMAP client capable of offline operation, and therefore employs a client-side cache.  This design choice directly
translates to its need of the UID mapping maintenance at all times.

\subsection{Synchronizing the UID Mapping}


