% vim: spelllang=en spell textwidth=120
\documentclass[trojita]{subfiles}
\begin{document}

\chapter{Acknowledgement}
\label{sec:acknowledgement}

This appendix provides full reference about third party contributions to the Trojitá code base.  It also mentions the
commercial ecosystem built around this open-source project.

\section{Commercial Applications}

\subsection{Partnership with KWest GmbH.}

In early 2010, I was contracted \cite{kwest-trojita} by a German embedded systems vendor, the KWest GmbH (lately
acquired by Blaupunkt).  That company was tasked by a German ISP for developing a tablet and were looking for an e-mail
client to ship.  Even though the tablet was not finished for reasons unrelated to Trojitá, but this collaboration
significantly improved the feature set of Trojitá.

\subsection{Collaboration with OpenMFG LLC, dba xTuple}

Later in 2010, an American CRM~\footnote{Customer Resource Management} vendor, the OpenMFG LLC, dba xTuple, were on a
search for a solution integrating customers' e-mail correspondence into their ERP database.  I was contracted to add the
required features to Trojitá.  The project successfully concluded in early 2011 and shipped on time.

More details about the whole architecture are available on xTuple's product website \cite{xtuple-trojita}.

\subsection{Nokia Developer Participation}

In 2011, my application for Nokia's Community Device Program was accepted.  Nokia was kind enough to provide their MeeGo
smartphone, the N950, on loan.  I have used this opportunity to build a version of Trojitá optimized for this platform
\cite{trojita-n950-preview}.

This application was also presented at the OpenMobility conference in Prague \cite{trojita-openmobility}.

\section{Third-party Contributions}

Although I'm the principal author of the vast majority of code in Trojitá, the project is run in an open environment as
a free software.  This model has attracted quite a few developers --- both individuals and from established software
vendors --- over the course of the project history.  This section presents a complete overview of all of their
contributions.  It is sorted in a chronological order.

\begin{description}
  \item[Justin J] contributed improvements to the GUI.
  \item[Benson Tsai] started the effort of providing an optimized user interface suitable for portable devices.
    Portions of his work remained unmerged due to technical issues in compatibility with the desktop version, but his
    patches were most inspiring.
  \item[Gil Moskowitz] from {\em OpenMFG LLC, dba xTuple} contributed patches towards better PostgreSQL integration in
    the xTuple e-mail synchronizer shipped as part of Trojitá.
  \item[John Rogelstad] from {\em OpenMFG LLC, dba xTuple} improved PostgreSQL interoperability through use of xTuple's
    own {\tt XSqlQuery} classes and fixed a build failure on Windows.
  \item[Jiří Helebrant] contributed GUI improvements and the Trojitá's logo and application icon.  He is also the author
    of the web site design.
  \item[Jun Yang] submitted a patch fixing a build failure under Visual Studio~2008.  He also reported an
    interoperability problem with {\tt STATUS} response parsing with servers not conforming to RFC~3501.
  \item[Andrew Brouwers] is the author of the {\tt .desktop} file which ships with Trojitá and of the initial version of
    the {\tt .spec} file used for building RPMs.
  \item[Tomáš Kouba] cleaned up the C++ code.  He also reported build failures with older versions of Qt.
  \item[Mariusz Fik] of {\em OpenSuSE} improved the {\tt .spec} file.  He also started using OpenSuSE's Open Build
    Service for building Trojitá.
  \item[Thomas Gahr] from {\em Ludwig-Maximilians-Universität München} implemented a simple address book, improved the
    GUI and made sure that new arrivals are properly reported.
  \item[Shanti Bouchez] added support for tagging e-mails with arbitrary keywords.  She also improved SMTP
    interoperability and enabled SSL/TLS support in there.
  \item[Chase Douglas] from {\em Canonical Ltd.} added a feature for hiding of already read messages from the message
    listing.
  \item[Wim Lewis] fixed encoding of human-readable names in outgoing messages.
  \item[Thomas Lübking] contributed a feature for raw IMAP searching.  He is also working on other improvements related
    to the {\tt QWidget} usage, IMAP searching and GUI fixes in general.
\end{description}

I'd like to use this opportunity to also extend my gratitude to all users who reported bugs or encouraged further
development of Trojitá.

\section{Use of Existing Libraries}

Trojitá makes use of the following third party libraries:

\begin{description}
  \item[The Qt framework] is used throughout the code as Trojitá is a Qt application.
  \item[The Qt Messaging Framework] provided code for wrapping the deflate compression algorithm in a Qt API.  It is
    also used for low-level character set conversions and MIME encoding/decoding.
  \item[ZLib] is used as a backend for actual deflate compression and decompression.
  \item[The KDE project's PIM libraries] was used for low-level string manipulation, character set conversion and
    related operations.
  \item[The QwwSmtpClient] library from Witold Wysota is used for speaking the SMTP protocol.  Several fixes were
    applied on top of the original release.
  \item[ModelTest] is a testing tool for verifying {\tt QAbstractItemModel} invariants. It is shipped as part of the
    source tree for technical reasons.
\end{description}

\end{document}
