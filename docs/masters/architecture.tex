% vim: spelllang=en spell textwidth=120
\documentclass[trojita]{subfiles}

\begin{document}

\chapter{Trojitá's Architecture}

This chapter provides a brief introduction to the architecture of Trojitá from a programmer's point of view.

\section{Overview of Components}

Trojitá makes heavy use of certain idioms common in Qt programming and in object-oriented software development in
general.

At the highest layer lies to GUI, graphical code managing the actual user interaction.  This code contains no knowledge
of IMAP or any other e-mail protocol; it is simply a graphical presentation layer specific to the desktop version of
Trojitá.  In the releases intended for mobile platforms, the traditional {\tt QWidget}-based GUI is replaced by a
variant built on top of Qt's QML \cite{qml}, a framework especially suited for touch-centric design.

Any interaction with IMAP is initiated through the model-view framework \cite{qt-mvc}.  A core class encapsulating a
representation of a single IMAP server, the {\tt Model} class, is accompanied by various proxy models and other
utilities to segregate and transform the data into better shape suitable for the upper layers.

Any action which shall take effect is, however, not performed by any of the model-related classes.  Trojitá utilizes the
concept of {\em tasks}, a set of single-purpose classes each serving a distinct role.  Examples of such tasks are
``obtain a connection'', ``synchronize a mailbox'', ``download a message'' or ``update message flags''.

One layer below the Tasks, the Parser is located.  This class along with its support tools converts data between a
byte stream from/to the network and higher-level commands and responses which are utilized by the upper
layers.  Actual network I/O operations are handled through a thin wrapper around Qt's own {\tt QIODevice} called {\tt
Streams}.~\footnote{The {\tt QIODevice} is wrapped to allow for transparent compression using the deflate algorithm.  Due
to the historic reasons, the {\tt Stream} subclasses use has-a over the is-a approach; this was required back when
Trojitá shipped IO implementations not based on the {\tt QIODevice} class.}

\subsection{Handling Input/Output}

The raw byte streams provided by network abstraction classes are intercepted by the {\tt Parser} class.  It takes any
incoming octet sequence and with help of the {\tt LowLevelParser} instantiates the concrete subclasses of the {\tt
AbstractResponse} class.  Actual parsing of the responses is typically deferred to the corresponding {\tt Response}
constructors which will tokenize the incoming data through {\tt LowLevelParser}'s methods and act on their contents
according to the corresponding grammar rules.

The response parsing had to be substantially relaxed from its original state due to observed interoperability issues.
Even the most popular IMAP implementations struggled with following the ABNF-defined \cite{rfc5234} syntax to the
letter; the most iconic example is Google's service which prevents the IMAP clients talking to it from accessing
forwarded messages \cite{gmail-bodystructure-sucks}.~\footnote{As of mid-2012, this issue remains unfixed.}

For some time, Trojitá's {\tt Parser} was implemented in a separate thread in an attempt to improve performance.
However, recent profiling shows that there amount of time spent in parsing is negligible compared to the rest of the
application with the only exception being the {\tt THREAD} response which essentially requires a complex transformation
of nested lists.  As such, the whole of Trojitá is now a single threaded application.~\footnote{The WebKit engine used
in HTML rendering creates threads for its individual purposes; the similar behavior is found in the QML engine used in
the mobile version.  These threads are not counted here, for they are considered to come from the ``system libraries''.}

\subsection{The Concept of Tasks}

The Tasks are designed to collaborate with each other, and there is a network of dependencies on how they can be used,
so that each task is responsible for only a subset of the overall work required to achieve a particular goal.  For
example, when a message shall be marked as read, an {\tt UpdateFlagTask} is created.  Because ``marking message as
read'' is implemented by manipulating IMAP message flags which can only happen while a mailbox is selected, the {\tt
Model} is asked to return an instance of a {\tt KeepMailboxOpenTask}, a ``manager'' class which is responsible for
keeping the mailbox state synchronized between the server and the client.  If the mailbox is already opened, an existing
instance of the {\tt KeepMailboxOpenTask} is returned; if that is not the case, a new instance is returned.  The {\tt
UpdateFlagsTask}'s execution is blocked and will commence only when (and if) the acquisition of a synchronized status
succeeds.  Similarly, this ``maintaining'' task itself deals just with incremental {\em updates} to the mailbox state,
actual mailbox synchronization is delegated to the {\tt ObtainSynchronizedMailboxTask}.  This synchronizer obtains the
underlying connection handle through consultation with the {\tt Model} to decide whether a new connection shall be
established or an existing one shall be re-purposed.  This policy decision is completely contained in the {\tt Model}
and is not visible to other layers at all.

Similar divisions of responsibility exist at other scenarios; the supporting infrastructure makes sure that no actions
are started unless their prerequisites succeeded.

The whole hierarchy is presented to the user in various shapes; the most basic is a ``busy indicator'' showing whether
any ``interesting'' activity is taking place.  A more detailed view showing an overview of all active tasks in a
tree-like manner illustrating their dependencies is available (see the {\tt TaskPresentationModel} class in Trojitá's
sources).

The introduction of Tasks to Trojitá in 2010 proved to be an excellent decision allowing for much increased development
pace.  Thanks to a proper use of git's branches, the transition was undertaken over a longer time pried without
disturbing the ongoing development.

The Tasks are instantiated by an auxiliary factory class in order to better facilitate automated unit testing with
mock objects.

\subsection{Routing Responses}

The Tasks introduced in the previous section are also used for proper response processing.  At all times, the {\tt
Model} can access a list of ``active tasks'' which are assigned to a particular IMAP connection.  When a response
arrives, it is dispatched to these tasks in a well-defined order until either of them declares the response as
``handled''.  If no Task claims responsibility for a particular response, the {\tt Model} itself takes action.  This
action might be either regular processing, or, in case the {\tt Model} cannot safely take care of said response, an
error is raised and the connection is sewered to prevent possible data loss.

The responses themselves are communicated to the {\tt Model} (and, through it, to the responsible Tasks) through a queue
of responses using Qt's own signal-and-slot mechanism.  The same way of passing data is used for additional
``meta information'' like informing about connection errors, parser exceptions or the low-level SSL/TLS certificate
properties.

\subsection{Models and Proxies}

Historically, much of the IMAP logic in Trojitá has been concentrated in the {\tt Model} class.  Over the time, I've
refactored that code to separate modules; however, even today, the {\tt Model} handles both data publication through the
Qt's model-view framework as well as IMAP response dispatch to the relevant Tasks.  With little less than two thousands
of physical lines of code, the {\tt Model} remains the second-largest source file in the code base (the biggest file is
a unit tests related to various modes of mailbox synchronization).

The {\tt Model} itself exports a view to anything available to a particular IMAP server account through Qt's model-view
API.  The actual data is stored in a tree formed by various {\tt TreeItem} subclasses.

As showing a single, rich-structured tree containing {\em everything} from mailboxes to individual message parts would
not be particularly user-friendly, a number of so called proxy models were created.  These models usually operate on
Qt's {\tt QModelIndex} level, but where profiling showed that a compelling speed increase would result from bypassing
the {\tt QVariant} abstraction, direct access through the underlying (and Trojitá-specific) {\tt TreeItem} classes was
used instead.  This has notably affected the design of the {\tt ThreadingMsgListModel} which is the biggest consumer of
the CPU time under the normal operation of the application.

Proxies exist for performing various transformations and filtering of the available data; some of them (like the {\tt
MailboxModel} and {\tt MsgListModel} are generic enough and used in the desktop client, a mobile application and the
single-purpose batch synchronizer (see appendix \ref{sec:xtuple}, p.~\pageref{sec:xtuple}), others (like the {\tt
PrettyMsgListModel}) are exclusive to the traditional desktop GUI.

\subsection{Lazy Loading and Cache}

The model-view separation strictly followed through Trojitá proved to be very useful when leveraging the full potential
of many advanced IMAP features.  Because the individual message parts are accessible separately, Trojitá's native mode
of operation supported the often-mentioned use case of ``only download a tiny text accompanying the big photo attachment
and then ask user whether to download'' without any effort.  At the same time, this separation made certain tasks which
were typically considered trivial a bit more demanding, e.g. when forwarding a message, Trojitá has to explicitly
retrieve two independent body parts and explicitly join them together.  On the other hand, in most of the situations
this separation brought benefits visible to the end user which trumped the minor, uncommon complications.

Any data once downloaded are kept in a persistent cache.  This feature allows Trojitá to work extremely well under
unfavorable network conditions.  It also allows its users to access any data which were already known previously in an
offline mode.

Several caching backends are shipped in Trojitá; some of them store data exclusively in memory and therefore are not
persistent per se, others use sqlite for actual data storage.  Another mode which offloads ``big data'' storage to
additional on-disk files is used by default.~\footnote{Work on the structured cache backends was sponsored by the KWest
GbmH. (appendix \ref{sec:kwest}, p.~\pageref{sec:kwest}).}

\section{The Mobile Version}
\todo[inline]{Also mention the version usable on cell phones}

\section{Regression Testing}
\todo[inline]{Time for a shameless plug about the test suite}
\todo[inline]{Mention the low-level optimizations --- string deduplication, results of profiling,\ldots}


\end{document}
