% vim: spelllang=en spell textwidth=120
\documentclass{book}
\usepackage[utf8]{inputenc}
\usepackage[czech,english]{babel}
\usepackage{a4wide}
\usepackage{listings}
\usepackage{longtable}
\usepackage{graphicx}
\usepackage{etoolbox}
\usepackage{lmodern}\renewcommand{\ttdefault}{lmvtt}
\usepackage[T1]{fontenc}
\usepackage{textcomp}
\usepackage{hyperref}
\usepackage{minted}
\usepackage{subfiles}
\usepackage{todonotes}

% there's no abstract environment in chapter
% http://tex.stackexchange.com/questions/3468/an-abstract-at-the-start-of-every-chapter
\newenvironment{abstract}{\begin{quote}\itshape}{\end{quote}}

\newcommand{\secref}[1]{section~\ref{#1} on page~\pageref{#1}}

\begin{document}

\title{Mobile~IMAP: Advanced IMAP Features in the Trojitá E-Mail Client}

\author{Bc. Jan Kundrát}

\begin{titlepage}
\maketitle
\end{titlepage}

% Now this is a marvelous memo from http://blog.haraldkraft.de/2012/04/imap-is-evil/:
% (Credit goes to Heidar Bernhardsson, http://iseld.org/):
% I also believe the only reason we still use email is that it’s impossible or very difficult to replace, kind of like Facebook in a way.

\setcounter{tocdepth}{3}
\tableofcontents
\subfile{acknowledgement}

\subfile{imap-protocol}
\subfile{extensions}

%\part{Additional Resources}
%\appendix
%\subfile{cd-structure}

%\bibliography{references}
%\bibliographystyle{iopart-num}

\end{document}
