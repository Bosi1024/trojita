% vim: spelllang=en spell textwidth=120
\documentclass[trojita]{subfiles}

\begin{document}

\chapter{IMAP Extensions}
\label{sec:imap-extensions}

\begin{abstract}
  Previous chapter has outlined the generic mode of operation of the IMAP protocol and provided an overview of what
  features are available.  In this chapter, we will talk about how to improve on the basic functionality through the
  optional extensions.
\end{abstract}

It might be concluded from the brief analysis presented in the previous chapter that a few feature of IMAP are rather
limiting in its real-world deployment.  Fortunately, the IMAP protocol includes built-in support of extensions which
allow rather substantial changes to the network communication.

\section{Optimizing the Protocol}

Before dwelling into more advanced topics like improving the synchronization performance or adding new features, let's
have a look at the basic layer of the IMAP protocol and investigate how these affect performance.

\subsection{The LITERAL+ Extension}
\label{sec:imap-literalplus}

One of the lowest-hanging optimization fruit to cater are IMAP's synchronizing literals.  In the basic IMAP, before a
clients proceeds with tasks involving upload of binary data (or any data over a certain size, for that matter), it has
to ask for an explicit server's approval based on the length of the data in question.  As we have shown previously, this
confirmation imposes a full round trip over the network, inducing latency and destroying any potential pipelining
improvements.

The LITERAL+ extension (RFC~2088~\cite{rfc2088}) simply lifts the requirement of having to wait for the server's
continuation requests by a subtle change of the syntax.  Adding an overhead of just one byte, the latency is completely
eliminated and communication gets rapidly streamlined.  We can go as far as to say that the code paths for dealing with
LITERAL+ data are actually simpler than having to deal with the old-fashioned synchronizing literals.  Consider the
following example:

\begin{minted}{text}
  S: * OK
  C: A001 LOGIN {11}
  # The client has to wait for server's response before proceeding any further
  S: + go ahead
  C: FRED FOOBAR {7}
  # A second round-trip wait occurs here
  S: + go ahead
  C: fat man
  S: A001 OK LOGIN completed
\end{minted}

When using the LITERAL+ syntax, the whole interaction happens without having to wait for the server:

\begin{minted}{text}
  S: * OK
  C: A001 LOGIN {11+}
  C: FRED FOOBAR {7+}
  C: fat man
  S: A001 OK LOGIN completed
\end{minted}

Trojitá includes full support for the LITERAL+ extension --- when it detects the {\tt LITERAL+} capability, it will
immediately switch to using non-synchronizing literals for increased performance.

\subsection{Data Compression}

IMAP is a textual, line-based protocol.  As such, it presents extremely good opportunities for compression --- using the
tried DEFLATE algorithm~\cite{rfc1951}, the basic IMAP chatter can be easily compressed to 25~-~40~\% of its original
size~\cite[p. 4]{rfc4978}.

RFC~4978~\cite{rfc4978} provides mechanism for exactly this functionality through the {\tt COMPRESS=DEFLATE} capability.
Trojitá ships with full support for this extension through the permissively licensed {\tt zlib} library.  Unfortunately,
the Qt's {\tt QSslSocket} currently doesn't provide a way to reliably tell whether the SSL connection is already
employing compression.  When combined with IMAP servers hidden behind SSL accelerators or load balancers (i.e. in
situations where the server does not have a clear idea whether the session is already compressed either), this has a
risk of needlessly trying to compress data twice.  This is a limitation in system libraries which cannot be overcome
without resorting to patching system components or conducting non-portable hacks.

\subsection{Improving Security through Cryptography}

RFC~2595~\cite{rfc2595} deals with best practices for establishing SSL/TLS connections to the IMAP server.  Trojitá
follows these recommendations, most notably it tries to establish a secure channel over {\tt STARTTLS} command even
without an explicit action on the user's side, should the server be configured to advertise itself as not accepting
logins over insecure connections through the {\tt LOGINDISABLED} capability.  A manual override is available in
situations where the SSL encryption is not available.

In advent of the recent breaches of many well-known (and widely trusted) Certificate Authorities~\cite{ssl-breaches},
Trojitá also comes with support for SSL key pinning~\cite{ssl-pinning}.  The trust model presented to the user is
similar to handling of SSH servers' public keys with OpenSSH --- upon first connection, the user is always presented with
a choice of whether to accept the certificate or not, along with a confirmation about whether the operating system and
its policy considers the certificate as ``trusted''.  No matter what the system-wide policy says, a changed public key
is always considered a threat and the situation is presented to the user accordingly.

\subsection{The IDLE Mode}
\label{sec:imap-idle}

We have mentioned that even though the protocol requires clients to be ready to accept any responses at any time, in
practice, servers are forbidden to send {\tt EXPUNGE}s when no command is in progress.  This requirement is necessary to
prevent a dangerous resynchronization as the server cannot possibly know whether the client has started to issue an
UID-less {\tt STORE} command which references messages through their sequence numbers.  Unfortunately, this directly
translates to clients having to {\em poll} the server quite often if they care about updates concerning the deleted
messages.

Any protocol which uses polling looks bad on paper --- having to poll leads to increased latency and higher power usage
because the equipment has to actively check for updates every now and then.  In contrast, {\em push-based} updates allow
the client to enter a low-power state where it merely waits to be woken up when a change occurs.  Such a mode is exactly
what the IDLE extension defined by RFC~2177~\cite{rfc2177} adds to IMAP.  It must, however, be said that real-world
concerns related to firewall timeouts and especially the NAT traversal has limited the usefulness of the IDLE command
somewhat, even to the extent where Mark Crispin, the original author of the IMAP protocol, claims that ``I see no
particular benefit to use of IDLE on a desktop machine''~\cite{crispin-idle-useless} --- a view which is not shared by
the wider community~\cite{tss-idle-keepalive} \cite{android-idle}, yet certainly worth a consideration.

The IDLE extension is basically a hack on top of the IMAP protocol which reverses a mantra of the basic IMAP
specification~\cite[p. 72]{rfc3501}:

\begin{quote}
  A command is not ``in progress'' until the complete command has been received; in particular, a command is not ``in
  progress'' during the negotiation of command continuation.
\end{quote}

With IDLE, a typical interaction might look like this one:

\begin{minted}{text}
  C: A004 IDLE
  S: * 2 EXPUNGE
  S: * 3 EXISTS
  S: + idling
  ...time passes; another client expunges message 3...
  S: * 3 EXPUNGE
  S: * 2 EXISTS
  ...time passes; new mail arrives...
  S: * 3 EXISTS
  C: DONE
  S: A004 OK IDLE terminated
  C: A005 FETCH 3 ALL
  S: * 3 FETCH (...)
  S: A005 OK FETCH completed
  C: A006 IDLE
\end{minted}

The whole effect of the {\tt IDLE} command is therefore to indicate to the server that the client is {\em really}
willing to listen for any updates to the mailbox state.  Because of compatibility concerns with legacy mail stores, the
IDLE extension still does {\em not} mandate the server to actually send updates about any changes as soon as they are
conducted --- indeed, a server which internally polls every fifteen minutes to check whether a message has arrived is
fully compliant with the IDLE extension, albeit rather useless to user who might expect (and, we might add, rightly so)
to be {\em instantly} notified about changes to the mailbox.

Trojitá includes full support for the IDLE extension and will enter that mode automatically shortly after a mailbox is
selected.  A simple heuristics is implemented which delays re-entering the IDLE command if it is likely that the
connection will be reused for any other purpose in near future, further eliminating needless data transfers.
Unfortunately, Trojitá is at the mercy of the IMAP server when it comes to superfluous data transfers, so it cannot
prevent the ``pings'' sent even when the connection does not contain a gateway with overly short timeouts.

\section{Improving Mailbox Synchronization}

The previous section dealt with optimizing the overall IMAP protocol as a whole.  At this stage, we can have a look at
more specific issues which cannot be easily overcome through generic measures like data compression using off-the-shelf
algorithms or updates to the basic protocol flows.

In the basic IMAP, neither the server not the client are required to keep any persistent state.  Clearly, it is
beneficiary for a client to keep downloaded copies of the immutable mailbox/message data (consult
\secref{sec:imap-immutable-data} in its persistent cache for some time, should the device constraints allow such a
storage.  There is still quite a lot of other data which has to be validated while the mailbox is being resynchronized.
Consider the following scenario where a mail user agent opens a mailbox with a thousand of message which has witnessed
expunges and new arrivals since the last time it was opened:

\begin{minted}{text}
  C: y1 SELECT foo
  S: * 1000 EXISTS
  S: * OK [UIDVALIDITY 12345] UIDs valid
  S: * OK [UIDNEXT 2345] Next UID
  S: y1 OK Selected
  C: y2 UID SEARCH ALL

    The following response has been shortened for demonstration purposes. In practice,
    it will have to contain a thousand of numbers.

  S: * SEARCH 2 4 5 6 7 8 9 10 ... 2290 2310 2311 2312 2333
  S: y2 OK Search completed
  C: y3 FETCH 1:1000 (FLAGS)
  S: * 1 FETCH (FLAGS ())
  S: * 2 FETCH (FLAGS (\\Seen))
  S: * 3 FETCH (FLAGS (\\Recent \$Answered))

  996 additional FETCH responses were omitted from this example for brevity.

  S: * 1000 FETCH (FLAGS (\\Seen))
  S: y3 OK fetched
\end{minted}

We can identify two steps which substantially contribute to the transferred data:

\begin{itemize}
  \item synchronizing the UIDs,
  \item updating flags.
\end{itemize}

In the rest of this section, we will have a look at optimization opportunities at each of these stages.  Please keep in
mind that some basic optimization heuristics concerning the UID synchronization were discussed in
\secref{sec:imap-mailbox-sync} along with reasons on why these steps are necessary in clients willing to maintain an
offline cache of immutable data.

\subsection{The ESEARCH Extension}

As seen in the protocol sample, the {\tt SEARCH} response containing UIDs of all messages in a mailbox can be rather
large.  At the same time, chances are that at least some of the adjacent messages might have been assigned contiguous
UIDs --- this is certainly not a requirement per se, but quite a few IMAP servers internally {\em do} assign UIDs from a
per-mailbox counter.  Real world, albeit anecdotal evidence \cite{cridland-uids-are-often-monotonic} indicates that this
scenario is very common, and therefore it might make sense to transmit the UIDs of all messages using the {\tt
sequence-set} \cite[p. 89]{rfc3501} syntax.  The ESEARCH extension, as defined in RFC~4731~\cite{rfc4731}, allows
exactly that:

\begin{minted}{text}
  S: * ESEARCH [TAG "y12"] UID 1,3:9,17:25,30:1000
\end{minted}

At the time of the ESEARCH adoption, the imap-protocol mailing list witnessed a disagreement on how exactly the {\tt
sequence-set} shall be interpreted.  Mark Crispin, the author of the original IMAP protocol (but not of the ESEARCH
extension) implemented ESERCH in a different manner.  He chose to take an advantage of the RFC3501-style definition of
UID sequences where the RFC mandates that servers shall treat non-existent UIDs given in sequence sets as if they
weren't referenced from the command at all.  For example, if the mailbox contained just UIDs 3, 5 and 10, a client using
the {\tt 3:10} construct has to be interpreted as if it requested {\tt sequence-set 3,5,10}.  Doing so present certain
optimization opportunities to the servers, for example when the client already knows the UID mapping and performs a
server-side search for messages matching certain criteria {\em and} the result set accurately matches an adjacent range
of messages, the server could take advantage of this adjacency a return a {\tt sequence-set} in the form of {\tt
10:150}, even though the mailbox contains only a few UIDs from this range~\cite{crispin-esearch-flawed}.  Furthermore,
his another point is that the clients already have two other ways of obtaining the UID mapping, either through the {\tt
UID SEARCH ALL} command or via an explicit {\tt UID FETCH 1:*}.  Needless to say, such a reasoning fails to take into
account potential bandwidth savings which can be rather substantial on ``reasonable'' mailboxes.  In the end, the
authors of the RFC 4731 disagreed with Crispin \cite{melnikov-esearch-interpretation}
\cite{cridland-esearch-interpretation}.

The ESEARCH extension allows nice bandwidth savings, so Trojitá tries to use it if the server says that it is supported.
In addition to that, format of the returned responses is changed so that it also includes the tag of the command which
caused it, allowing much more aggressive pipelining --- for example, clients are free to perform the UID discovery at the
same time as running a user-initiated search.  On the other hand, even in presence of ESEARCH, the UID mapping still has
to be synchronized explicitly.  This requirement is only lifted in the QRESYNC extension
(\secref{sec:extension-qresync}).  Before we describe that, though, it is necessary to have a look at the CONDSTORE.

\subsection{Avoiding Flags Resynchronization via CONDSTORE}

Leaving the UID synchronization alone for a while, let's have a look at various ways of eliminating the need to ask for
changed message flags.  In this case, no extension trying to reduce the data overhead of the {\tt FETCH} response was
proposed, but the problem got attacked from another side.

The whole point of flags synchronization is to be able to pick up changes which have happened since the last time was
selected.  If only the server was somehow able to assign a ``serial number'' to each change, we could subsequently ask
for all changes which have happened after a certain point.  The CONDSTORE extension from RFC 4551 \cite{rfc4551} works
in this way.

CONDSTORE-capable servers share a concept of ``modification sequence'', a {\tt MODSEQ}.  Each message in a mailbox is
assigned an unsigned 64bit integer.  Whenever message metadata (like its flags) change, the {\tt MODSEQ} of that
particular message gets increased.  Each increment is also required to reach a value which is higher than {\tt MODSEQ}
of any other message in that mailbox.  Similarly, a mailbox is assigned a {\tt HIGHESTMODSEQ}, an unsigned 64bit integer
which is interpreted as ``no message has ever had a {\tt MODSEQ} higher than this number'' --- of course subject to the
usual {\tt UIDVALIDITY} rules.~\footnote{As always, any change in {\tt UIDVALIDITY} directly translates to a full cache
flush and discarding any data previously remembered for the affected mailbox.  This includes not only the immutable data
of messages, but also the UIDs, message flags and --- in this extension --- the {\tt MODSEQ} and {\tt HIGHESTMODSEQ}
values.}

When a CONDSTORE-capable client opens a mailbox which was previously synced (and if the server supports CONDSTORE as
well, of course --- keep in mind that the IMAP extensions are strictly voluntary in their nature), at first it synces the
UID mapping as usual, possibly through the ESEARCH command discussed earlier.  After that, the client can use an
extended variant of the {\tt FETCH} command to ask for flags of those messages whose {\em current} {\tt MODSEQ} is
higher than the {\tt HIGHESTMODSEQ} which the client has remembered previously.  The server will respond with regular
{\tt FETCH} responses for each affected message.  In result, after this interaction is completed, the client is aware of
all pending flag changes and is fully resynchronized again.

This is how a typical synchronization might look like:

\begin{minted}{text}
  C: y1 SELECT foo (CONDSTORE)
  S: * 1000 EXISTS
  S: * OK [UIDVALIDITY 12345] UIDs valid
  S: * OK [UIDNEXT 2345] Next UID
  S: * OK [HIGHESTMODSEQ 715194045007]
  S: y1 OK Selected

    At this point, the client will obtain the UID mapping, likely through the UID SEARCH
    or its ESEARCH variant:

  C: y2 UID SEARCH RETURN () ALL
  S: * ESEARCH (TAG "y2") UID ALL 2,4:10,21:1008,2290,2310:2312,2333
  S: y2 OK Search completed

    At this point the CONDSTORE extension can be finally utilized -- only changed flags
    will be transmitted:

  C: y3 UID FETCH 1:1000 (FLAGS) (CHANGEDSINCE 613184045007)
  S: * 997 FETCH (UID 2310 MODSEQ 715194045007 FLAGS (\\Seen \\Deleted))
  S: * 1000 FETCH (UID 2333 MODSEQ 715194045005 FLAGS (\\Seen))
  S: y3 OK Fetched
\end{minted}

The algorithm is race-free --- as every message has a separate {\tt MODSEQ} counter, the delay between the {\tt SELECT}
and {\tt FETCH} command doesn't lead to data loss; by the time the {\tt FETCH} completes, the server guarantees that the
client have received any pending updates since the last synchronization.

The CONDSTORE is an extremely valuable extension; its savings on big mailboxes are predictable and automatic --- instead
of having to transmit $O(n)$ responses where $n$ is the number of {\em messages}, only $O(m)$ are required under QRESYNC
with $m$ being the number of {\tt modifications}.  This is an extension which, unfortunately, places a certain burden on
the IMAP server which has to track the serial numbers of messages' metadata; however, given the obvious reductions in
bandwidth, many servers have already implemented it, most notably the Dovecot and Cyrus open source IMAP servers.
Trojitá includes full support for this extension, making use of it whenever it is available.

\subsection{Optimizing UID Synchronization with QRESYNC}
\label{sec:extension-qresync}

The CONDSTORE extension discussed earlier has brought in the concept of a server-side state tracking and used that to
allow bandwidth-efficient way of synchronizing flag changes.  Given that a CONDSTORE-capable server already tracks
certain state, it might be worthwhile to somehow extend this state to cover the deleted messages as well.  As it turns
out, such a mechanism is implemented in the QRESYNC extension which is defined by RFC 5162 \cite{rfc5162}.

The basic idea behind QRESYNC is that as long as the UID mapping was fresh at some point in past, it is only necessary
to inform the client about which UIDs from that set are no longer there and push out the UIDs of newly arrived messages.

The QRESYNC extension modifies the {\tt SELECT command} so that it includes a few more parameters.  First of all, the
updated version of this command includes a tuple of {\tt (UIDNEXT, HIGHESTMODSEQ)} as known to the client.  If the {\tt
UIDNEXT} did not change, the server will have a look at the {\tt HIGHESTMODSEQ} value and in addition to essentially
behaving as a CONDSTORE server, it also sends out a list of expunged messages.~\footnote{Technically, the expunges are
sent out before the information about the updated flags, but that isn't the point here.}  A new response is defined for
this purpose, the {\tt VANISHED EARLIER}.

In format similar to the {\tt ESEARCH} response, the {\tt VANISHED EARLIER} contains a {\tt sequence-set} of UIDs which
the server believes that the client considers to be present in the mailbox.  Not only are these UIDs transmitted in a
compact syntax, thanks to the {\tt sequence-set} format, but the response typically contains only such UIDs which were
{\em just} removed.  The actual wording (and therefore the implications of this extension) are slightly different --- the
server is free to inform the clients about any UIDs, as long as they aren't in the mailbox right now, at the time of the
sync.  This is motivated by the need to relieve the servers from having to maintain a list of expunged UIDs
indefinitely, just in case a QRESYNC-enabled client reconnects after two years of inactivity.  When such a situation
happens, a server which cannot remember expunges going so far in history have no other option but to send a {\tt
VANISHED EARLIER} for {\em all} UIDs lower than the {\tt UIDNEXT}, no matter if they {\em ever} were present in the
mailbox.  This fallback suggests that the QRESYNC extension could very well have a negative net effect overall, at least
in certain pathological situations --- essentially when the list of expunges grows so long that the server decides to
prune some of its records.

In order to mitigate this issue, a few other options were added to QRESYNC.  The first of them is a way of indicating to
the server the range of UIDs about which the client actually cares.  The idea here is that if the client only cached a
subset of messages (for example those with UID higher than 50000), there isn't much point in informing the client about
each and every UID which might had been in the mailbox before (like those 49,999 of UIDs lower than 50,000).

However, chances are that this optimization is not enough to overcome the danger of having to sync too many UIDs --- and
indeed, some user agents might want to preemptively load messages both from the beginning {\em and} from the end of the
mailbox in an effort to optimize preloading.  Such user agents would not be able to benefit from the ``range of known
UIDs'' optimization.

Fortunately, the {\tt SELECT QRESYNC} command includes provisions for passing another type of data around --- it is also
possible to provide a representative list of $(sequence, UID)$ pairs.  Using technique similar to the proposed binary
search when discovering UIDs, the client can decide to send along a UID from roughly middle of the mailbox as the first
one, followed by another one located at circa 75~\% of the mapping, next one from $7 \over 8$ etc., halving the interval
in each step.  The concrete strategy to be pursuit is left to the client, as it is basically a policy decision.  Using
more fine-grained interval means that more data is sent along during each resynchronization without a direct merit (i.e.
when the server {\em still} remembers the previous {\tt HIGHESTMODSEQ} and can provide the client with relevant data),
while on the other hand sending less UIDs leads to minor data savings during ordinary reconnects while causing
potentially huge amounts of data to be transfered when the server is unable to use the --- perhaps very old or otherwise
stale --- {\tt HIGHESTMODSEQ}.

The presented version of Trojitá always uses halving of sequences, effectively transmitting $log_2(n)$ sequence-UID
pairs:

\begin{minted}{c++}
    Sequence knownSeq, knownUid;
    int i = oldUidMap.size() / 2;
    while (i < oldUidMap.size()) {
        // Message sequence number is one-based, our indexes are zero-based
        knownSeq.add(i + 1);
        knownUid.add(oldUidMap[i]);
        i += (oldUidMap.size() - i) / 2 + 1;
    }
\end{minted}

Our example synchronization (a thousand messages in a mailbox, unspecified changed since the last time) could therefore
look like this --- the client has fresh enough UID mapping up to UID 1008 (sequence number 995).  The server does not
remember deletions so far to the past, but the passed UID mapping fragment could nevertheless be used to optimize the
delivered data:

\begin{minted}{text}
  C: SELECT foo (QRESYNC 12345 613184045007 (500,750,875,937,968,983,990,995,997
  512,772,887,949,980,985,995,1002,1008,1111))
  S: * 1000 EXISTS
  S: * OK [UIDVALIDITY 12345] UIDs valid
  S: * OK [UIDNEXT 2345] Next UID
  S: * OK [HIGHESTMODSEQ 715194045007]
  S: * VANISHED (EARLIER) 1009:2289,2291:2309,2313:2332,2343,2344
  S: * 997 FETCH (UID 2310 MODSEQ 715194045007 FLAGS (\\Seen \\Deleted))
  S: * 1000 FETCH (UID 2333 MODSEQ 715194045005 FLAGS (\\Seen))
  S: y1 OK Selected
\end{minted}

The received data contain enough information to reconstruct complete UID mapping even under these unfavorable
conditions.  The redundant UID information in the {\tt FETCH} responses can be also used to make sure that both sides of
the connection arrive to the same final state --- and indeed, Trojitá will throw an error when it detects failure at any
time.

To further reduce the amount of data transmitted during the IMAP session, the QRESYNC extension also introduces a second
kind of the {\tt VANISHED} response --- the one without the {\tt EARLIER} modifier.  Serving as a substitute for the
ordinary {\tt EXPUNGE}, the {\tt VANISHED}'s biggest advantage is that it can inform about multiple expunges in a single
response.  Somewhat ironically, this modification also relieves the clients of their need to maintain a complete
UID-sequence mapping at all times --- but only after providing a method of making this synchronization severely less
painful in the first place.  The whole matter is complicated a bit more by the wording of the RFC which is pretty clear
on that the {\tt VANISHED} responses {\em should} be sent instead of {\tt EXPUNGE} --- a language which, in RFC terms,
means that the servers are supposed to do so, yet the clients are forbidden from relying on such behavior because under
special circumstances, the servers might very well have a good reason to defer back to the {\tt EXPUNGE}~\cite{rfc2119}.

Unfortunately, the combination of offset-based {\tt EXISTS} (which is a response used to inform the client about growth
of the number of messages in a mailbox) with UID-based {\tt VANISHED} can lead to races, a dangerous condition which we
have identified~\cite{kundrat-vanished-race}.  The problem lies in QRESYNC's allowance for non-existent UIDs to be
included in the {\tt VANISHED} response.  Consider the following scenario where the client is fully synced with a
mailbox with just a single message bearing UID 5.  The mailbox' {\tt UIDNEXT} is 11:

\begin{minted}{text}
  S: * 3 EXISTS
  C: x UID FETCH 11:* (FLAGS)
  S: * VANISHED 12:20
\end{minted}

The server is telling the client that any UIDs between 12 and 20 are gone.  The problem is that the client cannot
possibly know whether any message has got any of these particular UIDs, i.e. whether the messages \#2 and \#3 (the first
and second arrival) fall into that range.  Trojitá will immediately send out a request for UIDs of the new arrivals
(that is the {\tt UID FETCH} command in the previous example), but due to the timing issues, it is perfectly possible
that these messages are ``long'' gone (and the appropriate {\tt VANISHED} sent) by the time the server receives the {\tt
UID FETCH} command.  There isn't much a compliant IMAP client can do at this point besides issuing an explicit command
for finding out whether any new messages have actually remained in the mailbox.  This is a minor deficiency in the
QRESYNC extension which could be easily avoided by replacing the {\tt EXISTS} in manner similar to how {\tt EXPUNGE} got
replaced by {\tt VANISHED}.  The previous example would look like this one, eliminating any possibility of races:

\begin{minted}{text}
  S: * ARRIVED 12,33
  C: x UID FETCH 11:* (FLAGS)
  S: * VANISHED 12:20
\end{minted}

The {\tt ARRIVED} command is defined in a proposed draft in \secref{sec:draft-arrived}.  A similar functionality could
be achieved through the NOTIFY extension defined by RFC 5465 \cite{rfc5465}, but supporting NOTIFY is a rather strong
undertaking for an IMAP server and --- as of July 2012 --- support for that extension is still rather
scarce.~\footnote{None of the widely deployed open source IMAP servers supported {\tt NOTIFY} at the time this thesis
was written.  The Dovecot IMAP server had an experimental branch with partial support which would be enough to serve as
a replacement for {\tt ARRIVED}, but it could not be discovered because the {\tt NOTIFY} is an all-or-nothing extension;
it mandates full server support and doesn't include provisions for partial functionality which would have been enough in
this case.}

The QRESYNC extension also mandates an interesting mechanism for its activation.  Allegedly due to the backward
compatibility concerns with the {\tt VANISHED} response, clients have to support the ENABLE extension as well (RFC 5161
\cite{rfc5161}).  However, under a closer examination we can demonstrate that backward compatibility is actually not an
issue here --- compliant clients have to explicitly activate the QRESYNC extension by issuing a specially modified {\tt
SELECT} or {\tt EXAMINE} command, so it appears that using the {\tt ENABLE} command is not really necessary.  In fact,
based on the temporal proximity of RFC 5161 and 5162 (dealing with ENABLE and QRESYNC, respectively) and an overlap in
authorship of these documents, it appears that their designers simply wanted to make use of a mechanism which appeared
to be possibly useful in future.

Unfortunately, there is also a certain uncertainty about the {\tt ENABLE} command --- the errata \#1365 for RFC 5162
\cite{rfc5162-errata} proposes to add an explicit note that ``(A server MUST respond with a tagged BAD
response if) (\ldots) or the server has not positively responded to that command with "ENABLED QRESYNC", in the current
connection'', even though the RFC 5161 explicitly allows for aggressive pipelining of {\tt ENABLE} and {\tt
SELECT}.~\footnote{``There are no limitations on pipelining ENABLE.  For example, it is possible to send ENABLE and then
immediately SELECT, or a LOGIN immediately followed by ENABLE. \cite[p. 2]{rfc5161}}  We have raised this issue on the
{\tt imap-protocol} mailing list and the consensus there was that it is indeed allowed not to wait for the server's {\tt
ENABLED} before issuing a {\tt SELECT \ldots QRESYNC} \cite{melnikov-qresync-enable}.  We fully support such an outcome
as it would be rather awkward to see a requirement for extra network round trips in contemporary IMAP extensions.

\section{Fetching the Data}

The IMAP protocol contains rather rich set of features aimed at downloading the message data in an efficient manner;
clients can defer parsing of the MIME message parts~\cite{rfc2045} to the server and deal with the individual data
separately.  In spite of that, there are a few optimization opportunities which can drastically reduce the amount of
data to be transfered.

\subsection{The BINARY Extension}

Historically, e-mail messages could only contain English text, for which a 7-bit character set and the US-ASCII encoding
was adequate.  However, with the advent of ``multimedia'', a steady pressure had emerged, leading to the MIME standard
family.  Using MIME, complex tree-like structures can be embedded in e-mail messages and transmitted over the Internet
mail.  However, at the time these ere introduced, there was a real risk of not being able to transmit such complex
messages over traditional communication channels which were often only 7-bit safe.  Due to these backward compatibility
concerns, a few standard method of converting arbitrary data to a textual form were conceived under the name of {\em
Content-Transfer-Encoding}.

The two most common encoding schemes are Quoted-Printable~\cite[p. 18]{rfc2045} and Base64~\cite[p. 23]{rfc2045}, the
latter of which is especially suitable for converting arbitrary binary data to a 7-bit form.  However, it is clear that
mapping generic 8-bit data into 7-bit octets (and eliminating the appearance of certain ``magic'' characters in the
process) cannot possibly work without inflating the total size of the transferred data.  For the Base64
Content-Transfer-Encoding, the mapping converts 8-bit input (i.e. 256 values per octets) into a target alphabet of only
64 characters, imposing an overhead of roughly 33~\% compared to the raw binary form.  Whenever the MIME-encoded message
is transmitted, the amount of transferred data is therefore roughly one-third higher than strictly required.

RFC 3516 \cite{rfc3516} adds a feature to work around this limitation through the {\tt BINARY} extension.  When the
server supports this feature, clients can delegate the Content-Transfer-Encoding processing to the server and receive
the raw binary data.  Perhaps surprisingly, the {\tt BINARY} extension was not supported in Dovecot, one of the most
popular IMAP servers, at the time this thesis was written.  Nevertheless, Trojitá includes full support for this
extension and will automatically fetch data using the appropriate FETCH modifier in order to reduce the amount of data
to send over the network.

\subsection{Server-side Conversions via CONVERT}

Certain devices might have limitations which the sender might not have expected when she prepared the message.  For
example, a screen in a cell phone could have a very low resolution.  Unless the user really wants to see full details
after zooming eight times, it might make sense to reduce the resolution of that 22-megapixel $5760 \times 3840$ image
produced by Canon 5D~Mk.~III to fit on a $480 \times 800$ pixels screen of a high-end smart phone from 2012.  Even if
the user actually wants to see the real image, it might be worthwhile to offer an access to a lower-resolution version
for a quick preview.  This server-side conversion is what the {\tt CONVERT} extension from RFC 5259 \cite{rfc5259} enables.

Unfortunately, it appears that there are actually {\em no} publicly available servers which offer support for
server-side conversions and the most popular open source implementations have not expressed much interest when asked for
their future plans.  Due to this induced inability to test this feature for interoperability, Trojitá doesn't support
the {\tt CONVERT} extension at this point.

\subsection{Metadata Decoding}

IMAP requires compliant servers to support MIME message parsing and RFC~2822 header decoding.  One feature which is
notable absent, though, is support for server-side decoding of RFC~2047-formatted message headers and IMAP's {\tt
ENVELOPE} fields.  This shortcoming is partially addressed by two RFCs --- the already mentioned {\tt CONVERT} extension
mandates support for character set decoding and conversions of RFC~2822 message headers while an experimental RFC 5738
\cite{rfc5738} adds an ``UTF-8'' mode which switches all {\tt FETCH} commands to return the decoded Unicode data,
including the IMAP's {\tt ENVELOPE}.  Sadly, support for this extension is similar to what has been already said for the
{\tt CONVERT} and one can also safely claim that the possible data savings are minuscule, if any.  In addition, the
UTF-8 extension changes quite a lot of assumptions from the traditional IMAP protocol, to the extent that one could very
well propose yet another extension which would {\em just} enable access to the decoded {\tt ENVELOPE} and RFC 2822
header data.  No such extension exists at this point, though.

If the IMAP protocol was designed today, mandating a full-featured RFC~2047 decoder would be an obvious addition, but
with the legacy of the protocol history, a requirement to implement a client-side decoder anyway and no available server
support and the RFC being marked as {\em experimental}, Trojitá does not try to use the {\tt SELECT \ldots UTF8}
parameter.

\section{Updating Mailboxes}

Previous sections have introduced the existing extensions aimed at improving mailbox synchronization and data download.
In this part, we will talk about how to optimize access to an already selected mailbox and how to get updated
information about other mailboxes.

\subsection{Sorting Messages}

In a typical IMAP scenario, the server can access data for any message in a mailbox in a very cheap way.  This is in a
strong contrast to its clients which are typically connected over a network whose quality could leave much to be
desired.  It might therefore make sense to offload the data-intensive processing to the IMAP server and only send the
results to the client.

RFC 5256 \cite{rfc5256} adds support for server-side sorting.  Using these features allows clients to request the server
to sort a subset of messages using predefined sorting criteria like the message date, the time of its arrival, name of
the sender, contents of the subject field etc.  This feature set is subsequently augmented by RFC 5957 \cite{rfc5957}
which adds another sorting method which prefers the ``real name'' included in the e-mail addresses instead of using the
raw {\tt foo@example.org} format.  The results of the {\tt SORT} operation are transmitted in a format similar to the
{\tt SEARCH} response.  Trojitá includes full support for both of the mentioned RFCs.

This feature alone is a tremendous improvements over the traditional method where clients would have to download
envelope data for all messages in a mailbox and then sort the messages based on the obtained data.  However, clients
have no way of reusing the sort result when a new message arrives, mandating at least a limited support for client-side
sorting as well.  This limitation is mitigated only by the {\tt CONTEXT} family of extensions which is discussed later
in this chapter.

\subsection{Threads and Conversations}

RFC 5256 \cite{rfc5256} also includes support for organizing messages into threads.  The threading algorithm takes a
look at various pieces of message metadata like their subjects or the {\tt Message-Id}, {\tt References} and {\tt
In-Reply-To} headers and builds a tree of messages where children of a particular node represent messages which were
made as responses to the parent one.  This RFC specifies a few threading algorithms from almost useless one (the {\tt
ORDEREDSUBJECT} which groups together messages sharing a similar subject, effectively bundling unrelated items like
generic ``info'', ``inquiry'' etc. messages) to almost perfect ones (the {\tt REFERENCES} which works like the {\tt
ORDEREDSUBJECT}, but also takes the special machine-readable headers into account).  Sadly, even the {\tt REFERENCES}
algorithm only sorts the threads by the time stamp of the thread root and not by the latest message in a thread,
effectively ``hiding'' new responses deep in the mailbox history.  It also still looks at the message subjects,
potentially lumping unrelated messages together.

Both of these limitations are removed by the {\tt REFS} threading algorithm \cite{draft-ietf-morg-inthread}.  Despite
being still in the draft phase, the Dovecot IMAP server includes full support for it, and Trojitá will use it if the
{\tt THREAD=REFS} capability is advertised.  In absence of the {\tt REFS} algorithm, Trojitá degrades to {\tt
REFERENCES} and {\tt ORDEREDSUBJECT}, respectively.

Unfortunately, none of these extensions addresses the need to re-download the whole thread mapping when a new message
arrives; matters are in fact even more complicated than when sorting because a single arriving message could have a wide
ranging effect on the whole threading information.  This issue is addressed by the proposed ``extended {\tt INTHREAD}''
described in \secref{sec:draft-inthread-ext}.

\subsection{Incremental Sorting and Searching}

Soon after the original {\tt SORT} command got standardized, it became apparent that having to request a full update of
the sort order whenever a new message arrived would nullify some of the bandwidth saving opportunitites of performing
the server-side sort.  A similar concern existed for server-side searching --- here, too, the client would have to
explicitly check if the new arrivals match the search criteria.  This limitation was addressed by introduction of the
so-called {\em contexts} in RFC 5267 \cite{rfc5267}.

This RFC defines three extensions, the {\tt ESORT}, {\tt CONTEXT=SEARCH} and {\tt CONTEXT=SORT}.  The first one is
similar to the {\tt ESEARCH} (and in fact reuses the {\tt ESEARCH} response) in that it could reduce the amount of data
transmitted in response to the {\tt SORT} command.  The two {\tt CONTEXT=\ldots} capabilities extend the {\tt SEARCH}
and {\tt SORT} commands by a way to tell the server that it should tell the client whenever the results are updated.  An
efficient way of communicating the changes through the {\tt ADDTO} and {\tt REMOVEFROM} {\tt ESEARCH} return values was
introduced.

In absence of the search/sort contexts, a newly arriving message would typically result in client repeating the
operation.~\footnote{When speaking about searching, there is a pretty straightforward optimization opportunity where the
{\tt SEARCH} command can be augmented to include an explicit ``and the messages' UID is higher than the
$old\_uidnext$''.  Unfortunately, no such optimization can be performed for the {\tt SORT} command where the sort order
can possibly depend on each and every other message in a mailbox.}  This is turn leads to excess data transfers like in
the following example where a single arrival results in transferring the whole mapping again:

\begin{minted}{text}
  C: x1 UID SORT (SUBJECT) utf-8 ALL
  S: * SORT 1 2 10 5 50 20 ... [rest of UIDs goes here]
  S: x1 OK sorted
  ...time passes...
  S: * 1007 EXISTS
  C: x2 UID SORT (SUBJECT) utf-8 ALL
  S: * SORT 1 2 1111 10 5 50 20 ... [rest of UIDs goes here]
  S: x2 OK sorted
\end{minted}

For brevity purposes, we have omitted the sort order of the other 1,000 messages.  In presence of the {\tt CONTEXT=SORT}
extension, though, the same protocol interaction would look like this one:

\begin{minted}{text}
  C: x1 UID SORT RETURN (ALL UPDATE) (SUBJECT) utf-8 ALL
  S: * ESEARCH (TAG "x1") UID ALL 1:2,10,5,50,20,90:1090
  S: x1 OK sorted
  ...time passes...
  S: * 1007 EXISTS
  S: * ESEARCH (TAG "x1") UID ADDTO (2 1111)
\end{minted}

Using the sort contexts therefore leads to dramatic bandwidth savings on ``busy'' mailboxes which keep receiving new
mail over time; the {\tt ESEARCH} response can also occasionally contribute to a reduced amount of transferred data when
the UIDs were assigned in a favorable way.  The biggest contribution is, however, the introduction of the {\tt ADDTO}
{\tt ESEARCH} return response which obliterates the need to transfer sort order of the whole mailbox along with the
single updated item.

The search and sort contexts also impose certain, albeit abysmal overhead, though --- whenever a message is removed, an
explicit {\tt ESEARCH REMOVEFROM} has to be issued.  This funcitonality is mandated by the relevant RFC, even though it
doesn't provide any extra functionality --- all clients {\em still} have to listen for {\tt EXPUNGE} or {\tt VANISHED}
untagged responses (and rightly so, obviously), so there is no technical obstacle in requiring them to remove the
freshly removed items from their search/sort result cache.  Etiquette also dictates that clients shall cancel these
updates when they are no longer needed through the {\tt CANCELUPDATE} command.  These drawback are however extremely
small when working with larger mailboxes and absolutely worth the increased benefit of not transferring the whole set of
UIDs over and over again.

Trojitá includes full support for both {\tt CONTEXT=SEARCH} and {\tt CONTEXT=SORT}.  The {\tt CONTEXT=SEARCH} has been
tested for interoperability against the Dovecot IMAP server, one of the few implementations which actually offer this
functionality.  Unfortunately, we were not able to locate a single IMAP server supporting {\tt CONTEXT=SORT}, so no
real-world interoperability tests could have been performed.~\footnote{As any other feature of Trojitá, though, the
support for {\tt CONTEXT=SORT} is covered by the automated test suite which was carefully written to minimize the chance
of any unwanted side effects and guard against unintended results when the servers deviate from the expected behavior.}
This is a rather surprising outcome given that the RFC 5267 is not exactly a fresh standard now and that its authors
work from a commercial software vendor offering a pretty advanced IMAP server implementation.

Sadly, no equivalent of these incremental updates is defined for the {\tt THREAD} command.  We suspect that this is
caused by the fact that a single new arrival can affect each and every message in the previously received thread
mapping.  We have attempted to address this limitation by augmenting the {\tt SEARCH=INTHREAD} draft, see
\secref{sec:draft-inthread-ext} for details about the selected approach.

\subsection{Advanced Searching}

\todo[inline]{write me}

FUZZY: \cite{rfc6203}

MULTISEARCH: \cite{rfc6237}

\subsection{Obtaining Statistics for Other Mailboxes}

Many IMAP clients start their session by requesting a {\tt LIST} of all top-level mailboxes.  This command is then
followed by a {\tt STATUS} for each of them in order to obtain information like the number of messages in each mailbox.
The obtained information is typically used in a GUI of some kind to show the mailbox list to the user.

The basic IMAP specification unfortunately doesn't convey the critical piece of information about whether a mailbox
contains any child mailboxes --- a data point typically required by any GUI to be able to show a proper widget for
opening a list of these child items.  In absence of the {\tt CHILDREN} extensions, as defined by RFC 3348
\cite{rfc3348}, clients have no choice but to issue an explicit {\tt LIST} command for all mailboxes,~\footnote{The only
exception being those marked with the {\tt {\textbackslash}NoInferiors} which is meant to indicate that this mailbox
could {\em never} contain any child mailboxes, perhaps due to technical limitations on the server side --- a very
different case from not containing any child mailboxes at {\em this} time.} trying to list their children.

The {\tt CHILDREN} is a pretty straightforward extension which shall arguably be supported by any IMAP server worth its
salt; its absence does not improve the situation in any way.  Trojitá supports it fully and will gracefully fall back to
extra {\tt LIST} requests in case it is not available.

The initial discovery of mailboxes also mandates a separate {\tt STATUS} command for each mailbox --- behavior which
arguably goes against the spirit behind that command which was intended to {\em not} serve as a generic ``tell me about
updates to other mailboxes'' feature.  This initial idea no longer has its merit, unfortunately --- users simply {\em
expect} being able to see a number of unread messages right next to the mailbox name, and client authors have to deliver
this information to them.  Having to send an extra {\tt STATUS} command for each mailbox during the initial discovery is
not too evil thing per se (the total wasted bandwidth is negligible when compared to mailbox synchronization), but worth
optimizing anyway.  RFC 5819 \cite{rfc5819} adds an option to the {\tt LIST} command to request automatic sending of the
untagged {\tt STATUS} command along.  When the IMAP server announces the {\tt LIST-STATUS} capability, Trojitá will
automatically make use of this extension.

\subsection{Push-notification of Other Mailboxes' State}

A feature most notably absent from the {\tt IDLE} is any support of passively monitoring changes of non-selected
mailboxes.  Over the time, many extensions have appeared in the state of various drafts
\cite{draft-wener-lemonade-clearidle} \cite{draft-gulbrandsen-imap-nostore} \cite{draft-magicaltux-imap4-idleplus},
often simply requesting unsolicited delivery of {\tt STATUS} and {\tt FETCH} responses.  None of these extensions gained
widespread support nor reached the state of a proposed standard, though.

Said status was reached by the {\tt NOTIFY} extension codified in RFC 5465 \cite{rfc5465}.  It adds an impressive amount
of features; compliant agents can listen for creation and deletion of mailboxes (eliminating the need to redo a
top-to-bottom {\tt LIST} discovery), changes in amount of messages in specified list of mailboxes and even for their
flag changes.  Sadly, support for this extension is scarce among the existing IMAP servers and its author reportedly has
mixed feelings \cite{arnt-good-bad-rfc} about its fate where basically ``noone implements it''.  In early 2012, a
posting on the Dovecot mailing list announced \cite{dovecot-imap-notify} preliminary support for a part of this
extension in Dovecot, one of the most widely deployed open source IMAP daemons.  Unfortunately, this code was not
complete as of July 2012 \cite{dovecot-hg-notify} and the branch we have tried contained regressions which prevented
regular use of other features of the IMAP server altogether.~\footnote{This is not to say that the Dovecot is a bad IMAP
daemon --- not at all.  The version we have tried is a development snapshot which is clearly marked as an experimental
version which requires future work.}  Due to not being able to verify the {\tt NOTIFY} operation against any available
IMAP server implementation, Trojitá will not try to leverage this extension for the time being.

\section{Composing and Delivering Mail}

\todo{Write me}

CATENATE \cite{rfc4469}

\section{Further Improvements}

Many additional extensions have been defined over years, covering various areas of the protocol.  This section deals
with those extensions which do not quite fit into any of the previous categories.

\subsection{Debugging}

Most of other communication protocols contain a way of letting the other party know what software implementation and in
which version it is talking to.  In the web, this is usually accomplished through the {\tt User-Agent} and {\tt Server}
headers, e-mail messages often contain either an {\tt X-Mailer} or {\tt User-Agent} field in theirs RFC 2822 headers,
etc.  IMAP adds a similar feature through the {\tt ID} extension defined in RFC 2971 \cite{rfc2971}.

This RFC is pretty clear on that implementations are explicitly {\em forbidden} from using {\em any} knowledge obtained
through the {\tt ID} extension to alter their behavior.  This is a reasonable decision intended to prevent clients
implementing blacklists and whitelists of ``known'' servers.  All IMAP protocol speakers are intended to only use the
{\tt CAPABILITY} responses (and the {\tt ENABLE} extension, if present) to change their behavior.  Trojitá is fully
compliant with these requirements.

\subsection{Internationalization}

The ``internationalization'' RFC (RFC 5255 \cite{rfc5255}) is focused on server implementations, mainly specifying how
to perform search/sort collation under various circumstances.  That said, it also presents two commands to the IMAP
clients.  The first of them is the {\tt LANGUAGE} command suitable for changing the language in which various error and
notifying messages are generated and sent by the server.  Trojitá tries hard to rely on machine-readable response codes
(RFC 5530 \cite{rfc5530}) instead.  The second command is the {\tt COMPARATOR}, a feature intended to let clients
specify which comparators and collators to use when performing search or sort operations (these comparators are defined
in RFC 4790 \cite{rfc4790}).  By default, servers supporting at least the {\tt I18NLEVEL=1} extension are required to
perform collations using the {\tt i;unicode-casemap} comparator \cite{rfc5051} --- a feature which is very useful (and
often sufficient) in countries using the Latin alphabet.  Those servers which support {\tt I18NLEVEL=2} also accept
client-specified preference about how to perform these operations.  Adding support for this higher level would be
trivial on a technical front (the {\tt LANGUAGE} and {\tt COMPARATOR} commands are very simple with much more demanding
requirements on servers), but we have received no requests for such a feature yet --- suggesting that the {\tt
i;unicode-casemap} comparator works well for most users at this point.

There is also the experimental ``UTF-8'' RFC \cite{rfc5738} whose aim is to get rid of any non-UTF8 data being
transferred over IMAP.  Unfortunately, as designed, this document presents backward-incompatible changes to the IMAP
protocol and is hence not widely supported by the common server implementations.  The client authors would very much
prefer to stop dealing with various Unicode encoding schemes, but this RFC does not completely address all of the
issues.

\subsection{Other Supported RFCs}

Trojitá is fully conforming to the description of the ``Distributed Electronic Mail Models in IMAP4'', as defined in RFC
1733 \cite{rfc1733}.  It also respect the recommendations about concurrent access to mailboxes from RFC 2180
\cite{rfc2180}, generic suggestions to the IMAP implementors (RFC 2683 \cite{rfc2683}), Mark Crispin's famous ``Ten
Commandments of How to Write an IMAP client'' \cite{crispin-ten-commandments} and Timo Sirainen's suggestions for client
authors \cite{tss-client-coding-howto} \cite{imapwiki-client-best-practices}.

It will make use of the {\tt UNSELECT} command for internal technical reasons (RFC 3691 \cite{rfc3691}), if available;
if it isn't present, it will gracefully revert to using fabricated mailbox names with the {\tt EXAMINE} command from the
baseline IMAP specification.  This failover is thoroughly verified by a suite of automated unit tests.

Trojitá is capable of recording responses from the {\tt UIDPLUS} extension (RFC 4315 \cite{rfc4315}); the author have
also contributed \cite{jkt-move-uidplus} to the discussion related to the {\tt COPYUID} equivalent for the proposed {\tt
UID MOVE} command.

The code also includes full support for recognizing various extensions to the IMAP's grammar (RFC 4466 \cite{rfc4466}),
response codes (RFC 5530 \cite{rfc5530}) and the reserved set of keywords (RFC 5788 \cite{rfc5788}).

RFC 5258 \cite{rfc5258} is used when available to explicitly encourage the IMAP server to send additional metadata in
{\tt LIST} responses --- the biggest benefit of this extension is eliminating the need of sending explicit {\tt LSUB}
requests to discover mailbox subscriptions.

Finally, the Lemonade extension (RFC 5550 \cite{rfc5550} and the obsolete RFC 4550 \cite{rfc4550}) has compiled a set of
requirements crucial to the mobile IMAP e-mail clients.  Comparison of the proposed set of extensions with the Lemonade
profile is presented in a dedicated section of this thesis on page \pageref{sec:lemonade-comparison}.

\subsection{Out-of-scope Features}

The extensions we have presented so far all have a certain affinity towards the ``mobile IMAP''.  Many other extensions
have been introduced, though, often solving a real problem.

The first family of extensions with debatable merit are extensions providing support for so-called referrals.  The RFC
2221 \cite{rfc2221} adds a mechanism for servers to redirect clients based on their identity, a feature which was
originally supposed to come handy in large corporate environment.  Similar to that, the RFC 2193 \cite{rfc2193} adds
mailbox referrals, a feature where a subset of user's mailboxes might be stored on a remote server.  As it happens,
these features have not attained a big market share among client developers \cite{thunderbird-referrals} and the servers
which supports that are generally willing to act as a transparent proxy for their clients anyway
\cite{cyrus-referral-proxy}.

Extensions which are useful in a general-purpose e-mail client are the {\tt NAMESPACE} extension (RFC 2342
\cite{rfc2342}) which would allow compliant clients to automatically discover where e.g. other users' mailboxes are
located, support for managing access-control lists (ACLs) on the server (RFC 4314 \cite{rfc4314} and its obsolete form
given in RFC 2086 \cite{rfc2086}) and finally support for reporting and managing storage quotas (RFC 2087
\cite{rfc2087}).  These have not been implemented in Trojitá yet.

A mechanism for increasing interoperability with organizations which have invested in a single-sign-on infrastructure
like Kerberos could be improved through better support for SASL (RFC 1731 \cite{rfc1731}, RFC4959 \cite{rfc4959}).  We
have already received a request from users which would very much prefer to have a GSSAPI-enabled Trojitá, unfortuntely
we have not been able to accommodate them due to time constraints.

A few extensions might improve the general comfort of users setting up their e-mail clients for the first time.  Without
any doubt, autoconfiguration through the DNS SRV records (RFC 6186 \cite{rfc6186}) falls into this category.

There are also certain features which might add a whole new level of functionality to working with e-mail -- examples
are the {\tt ANNOTATE} extension (RFC 5257 \cite{rfc5257}) for adding arbitrary annotations to individual messages or
even their parts, which is still marked as experimental, or the similar {\tt METADATA} feature (RFC 5464 \cite{rfc5464})
adding the same functionality on a server or mailbox level.  Needless to say, support for these extensions is scarce
among the IMAP clients.

Other extensions try to fill a certain niche.  Examples of these are the {\tt MULTIAPPEND} extension (RFC 3502
\cite{rfc3502}) which allows the client to atomically upload many messages at once.  Having such a feature could be a
terrific boon in clients which support batched import of data from the existing mail store, but it is not so valuable in
a generic client.  The {\tt WITHIN} extension (RFC 5032 \cite{rfc5032}) allows clients to search among messages of a
certain age; the {\tt SEARCHRES} (RFC 5182 \cite{rfc5182}) adds a low-level pipelining optimization which would allow
the client to re-use the previous search result in the subsequent commands.  RFC 5466 \cite{rfc5466} adds support for
persistent storage of search criteria on the server through the already mentioned {\tt METADATA} extension.  We have not
found a use case for having these optional extensions utilized from Trojitá in any place.

The RFC 3503 \cite{rfc3503} deals with how to generate the message delivery notifications (MDNs) in IMAP.  This document
clearly does not apply to clients which on purpose do not create MDNs for privacy reasons, such as Trojitá.

Finally, certain extensions improve the user experience in specialized environments.  One of them is RFC 5616
\cite{rfc5616}, an extension aimed at ``Streaming Internet Messaging Attachments''.  One could imagine a use case where
a carrier-level voice mailbox was implemented over IMAP; in similar situations, such a solution would have its merit.
At the same time, this specific extension has so special requirements on the network architecture that it is clearly
out-of-scope for a general-purpose e-mail client merely running on a cell phone.

\section{Obsolete Extensions}

IMAP is a rather old protocol (its history is slightly older than the author of this thesis, for that matter).  Certain
features have been therefore deprecated over time and it took years to grow to the current IMAP4rev1 version from the
old standards of IMAP2 \cite{rfc1064} \cite{rfc1176}, IMAP3 \cite{rfc1203} and IMAP4 \cite{rfc1730}.  Attention has been
paid to make this transition as smooth as possible through various compatibility recommendations \cite{rfc1732}
\cite{rfc2060} \cite{rfc2061} \cite{rfc2062}.

Trojitá requires the server to at least announce the {\tt IMAP4rev1} capability.  This protocol revision is currently
defined by RFC 3501 \cite{rfc3501} from March 2003; however, changes since the previous version from December 1996
(\cite{rfc2060}) are mostly backward compatible --- and in practice, we have received no report of Trojitá not being
able to work against any IMAP server implementation out there.

The {\tt UIDPLUS} extension got redefined from its former shape \cite{rfc2359} to the current revision \cite{rfc4315};
the new document states that the reason for this revision was to prevent sending of bogus UID replies when the target
mailbox did not support persistent UIDs.  As such, Trojitá can deal with both revision of the {\tt UIDPLUS} document.

\end{document}
