% vim: spelllang=en spell textwidth=120
\documentclass[trojita]{subfiles}

\begin{document}

\chapter{IMAP Extensions}
\label{sec:imap-extensions}

\begin{abstract}
  Previous chapter has outlined the generic mode of operation of the IMAP protocol and provided an overview of what
  features are available.  In this chapter, we will talk about how to improve on the basic functionality through the
  optional extensions.
\end{abstract}

It might be concluded from the brief analysis presented in the previous chapter that a few feature of IMAP are rather
limiting in its real-world deployment.  Fortunately, the IMAP protocol includes built-in support of extensions which
allow rather substantial changes to the network communication.

\section{Optimizing the Protocol}

\subsection{The LITERAL+ Extension}

One of the lowest-hanging optimization fruit to cater are IMAP's synchronizing literals.  In the basic IMAP, before a
clients proceeds with tasks involving upload of binary data or anything over a certain size, it has to ask for an
explicit server's approval based on the length of the data in question.  As we have shown previously, this confirmation
imposes a full round trip over the network, inducing latency and destroying any potential pipelining improvements.

The LITERAL+ extension simply lifts the requirement of having to wait for the server's continuation requests by a subtle
change of the syntax.  Adding an overhead of just one byte, the latency is completely eliminated and communication gets
rapidly streamlined.  We can go as far as to say that the code paths for dealing with LITERAL+ data is actually simpler
than having to deal with the old-fashioned synchronizing literals.  COnsider the following example:

\end{document}
